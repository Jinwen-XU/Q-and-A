\documentclass[%11pt,
  use style = classical,
  scroll,
]{Q-A}

\def\PackageVersion{2023/11/01}
\def\PackageSubVersion{}

\newcommand{\QApackage}{\normalfont\textsf{Q-A}}


\title{\QApackage{}\\\smallskip\itshape Typesetting Q\&A-style conversation made easier}
\author{Jinwen XU}
\thanks{Corresponding to: \texttt{\QApackage{}~\PackageVersion\PackageSubVersion}}
\date{\TheDate{\PackageVersion}[only-year-month], in Paris}


\newcommand{\meta}[1]{$\langle${\normalfont\itshape#1}$\rangle$}
\def\textcolon{:}
\def\texteqsign{=}
\def\textbacktick{`}
\def\textast{*}
\def\textsharp{\#}


\begin{document}

\maketitle


##+ {Introduction}

?
  What is this?

:
  \QApackage{} is a \LaTeX{} document class for you to typeset Q\&A-style conversation. It turns a simple pure text Q\&A dialog like this:

  == {code/QA-doc-code-sample-content.tex}

  into a carefully designed document like this:

  \begin{center}
    \fbox{\includegraphics[width=.67\textwidth]{code/QA-doc-code-sample-content-result.pdf}}
  \end{center}


##+ {Preparation}

?
  That is nice. How can I use it? Is there anything that needs to be prepared?

:
  You should make sure that this document class is properly installed.

  If you are using TeX Live 2024 or newer, or the most recent version of MikTeX, then this package should already be included, and you don't need to do anything.

  Otherwise, you need to check for package update to see if you can receive it. In case not, you can always go to \href{https://ctan.org/pkg/Q-A}{the CTAN page} to download the `.zip` file with all related files included.


##+ {Usage}

?
  Now that I have successfully installed it, could you propose an example of usage?

:
  Of course. A typical document looks like this:

  == [latex] {code/QA-doc-code-sample-document.tex}

  The available class options include:
  \begin{itemize}
    \item `scroll`: turns the scroll mode on, which generates a single-page pdf similar to a long screenshot. It is recommended to use this option if your document contains some large piece of code.
    \item `use theme = \meta{theme}`: use the selected theme, available choices include: `default` (like the current document), `ChatGPT-light` and `ChatGPT-dark` (see the demo documents).
    \item Font size options such as `11pt`, `12pt`.
  \end{itemize}

?
  What about the main content?

:
  You have already seen an example of the main content. As you might have noticed, there are several syntaxes. Let me explain.

  [Questions (Q), Answers (A), and Narrations (N)]
  \begin{itemize}
    \item A question begins with the prefix `Q:` or `?` (or `?`).
    \item An answer begins with the prefix `A:` or `:` (or `:`).
    \item A narration begins with the prefix `N:` or `"` (or `“`, `”` or `「`).
  \end{itemize}

  [Emphasize and Bold]
  \begin{itemize}
    \item Use `\textast\meta{text}\textast` to emphasis `\meta{text}`.
    \item Use `\textast\textast\meta{text}\textast\textast` to make `\meta{text}` into boldface.
    \item Use `\textast\textast\textast\meta{text}\textast\textast\textast` to combine the previous effects.
  \end{itemize}

  [Emphasized Enumeration]
  An emphasized enumeration, like the current line, is marked by `[\meta{text}]` at the beginning, where `\meta{text}` is the text to be emphasized.

  [Sections]
  \begin{itemize}
    \item You may start a new (*unnumbered*) ---
    \begin{itemize}
      \item section, via `\textsharp\textsharp{} \{\meta{section title}\}`;
      \item subsection, via `\textsharp\textsharp\textsharp{} \{\meta{subsection title}\}`;
      \item subsubsection, via `\textsharp\textsharp\textsharp\textsharp{} \{\meta{subsubsection title}\}`;
    \end{itemize}
    \item If you wish to use the *numbered* version, write `\textsharp\textsharp+`, `\textsharp\textsharp\textsharp+` and `\textsharp\textsharp\textsharp\textsharp+` instead.
  \end{itemize}

  [Input/Include Files]
  \begin{itemize}
    \item Use `\textcolon\textcolon{} \{\meta{file name}\}` to input a file.
    \item Use `\textcolon\textcolon\textcolon{} \{\meta{file name}\}` to include a file.
  \end{itemize}

  [Code]
  Due to the current implementation of this document class, it is unfortunate that you cannot directly insert source code in your document. There are some workarounds, though.
  \begin{itemize}
    \item For *displayed* code, stored the code into a separate file, and then use `\texteqsign\texteqsign{} \{\meta{file name}\}` to print it. You may also use an optional argument like `\texteqsign\texteqsign{} [\meta{language}] \{\meta{file name}\}` to select the language of your code.
    \item For inline code, you may simply write it between two backticks `\textbacktick\meta{code}\textbacktick`, similar to the Markdown syntax. However, be aware that special characters need to be escaped, for example, `\textbackslash` should be written as `\textbackslash textbackslash`, `\{` should be written as `\textbackslash\{`, `\%` should be written as `\textbackslash\%`, etc.
  \end{itemize}

  And don't forget that you are still using \LaTeX, so images, tables and lists can be written as usual.

?
  I see. Is there anything else for me to be careful about?

:
  Glad that you asked. Here are several things that should be taken care of:
  \begin{itemize}
    \item A question, answer or narration should always begin in a new paragraph.
    \item An emphasized enumeration should also begin in a new paragraph.
    \item Likewise, a `section`/`subsection`/`subsubsection` should be placed in a separate paragraph.
    \item Input or inclusion of files should also be operated in a separate paragraph.
  \end{itemize}

##+ {Get Support}

VOID
\bigskip

If you run into any issues or have ideas for improvement, feel free to discuss on:
\begin{center}
    \url{https://github.com/Jinwen-XU/Q-A/issues}
\end{center}
or email me via \href{mailto:ProjLib@outlook.com}{\texttt{ProjLib@outlook.com}}.


\vspace{3\baselineskip}

% ---

% "
%   Below is the code of the current document.

%   == [latex] {\jobname.tex}

\end{document}
